
\documentclass[12pt]{article}

% Packages
\usepackage[spanish,activeacute,es-tabla]{babel}
\usepackage{hyperref}
\usepackage{sectsty}

\sectionfont{\normalsize}

\title{Sistemas Digitales - 1era Guia Practica - Representación de la información}
\author{Román López Garro}


\begin{document}
\maketitle

\begin{abstract}
    Ejercicios resueltos de la primera guia, lo relevante es entender como podemos representar números en distintas bases y que limitaciones se tienen al utilizar ciertas interpretaciones de números en binario para manejar a los números positivos y negativos.
\end{abstract}

\tableofcontents

\newpage

% -------------------------
% Ejercicio 8
% -------------------------
\section*{Ejercicio 8}
\addcontentsline{toc}{section}{Ejercicio 8}
\phantomsection

\textit{¿Cómo acomodaría esta suma de números hexadecimales de 4 dígitos en notación complemento a 2, para que en ningún momento se produzca \textit{overflow}?}

\[
7744_{16} + 5499_{16} + 6788_{16} + \mathit{AB68}_{16} + \mathit{88BD}_{16} + 9879_{16} = 0003_{16}
\]

\textbf{Solucion: } \\
                    \\
                    Lo relevante de este punto es no perder de vista la cantidad de bits disponibles para operar. En este caso como tenemos \textbf{4 dígitos} en hexadecimal nos quedan \textbf{16 dígitos} máximos en binario, por lo tanto interpretados complemento a 2 algunos números van a ser positivos y otros negativos, los que nos hace muy sencillo saber en que orden se tienen que sumar puesto que simplemente podemos sumar positivos con negativos.

Primero convierto todos los numeros a binario, esto es independiente claramente del tipo de interpretación:

\begin{itemize}
    \item[] $7744_{16} = 0111 0111 0100 0100_{2}$
    \item[] $5499_{16} = 0101 0100 1001 1001_{2}$
    \item[] $6788_{16} = 0110 0111 1000 1000_{2}$
    \item[] $AB68_{16} = 1010 1011 0110 1000_{2}$
    \item[] $88BD_{16} = 1000 1000 1011 1101_{2} $
    \item[] $9879_{16} = 1001 1000 0111 1001_{2}$
\end{itemize}

Luego agarro un numero positivo (7744) y le sumo uno negativo (AB68): \\ \\

\begin{tabular}{c c c}
\texttt{7744}                        & \texttt{= 0111011101000100} \\
\texttt{AB68}                        & \texttt{= 1010101101101000}\\
\texttt{(7744 + AB68 + 9879 + 5499)} & \texttt{= 0010001010101100}
\end{tabular} \\ \\

Le sumo uno negativo, porque se que no me va a dar overflow. \\ \\

\begin{tabular}{c c c}
\texttt{(7744 + AB68)}        & \texttt{= 0010001010101100} \\
\texttt{9879}                 & \texttt{= 1001100001111001}\\
\texttt{(7744 + AB68 + 9879)} & \texttt{= 1011101100100101}
\end{tabular} \\ \\

Le sumo uno positivo porque se que no me a dar overflow, porque mi suma me dio negativo. \\

\begin{tabular}{c c c}
\texttt{(7744 + AB68 + 9879)}       & \texttt{= 1011101100100101} \\
\texttt{5499}                       & \texttt{= 0101010010011001} \\
\texttt{(7744 + AB68 + 9879 + 5499)}& \texttt{= 0000111101111110}
\end{tabular} \\ \\

Le sumo uno negativo de los numeros restantes: \\ \\

\begin{tabular}{c c c}
\texttt{(7744 + AB68 + 9879 + 5499)} & \texttt{= 0000111110111110} \\
\texttt{88BD}                        & \texttt{= 1000100010111101} \\
\texttt{(7744 + AB68 + 9879 + 5499 + 88BD)} & \texttt{= 1001100001111011}
\end{tabular} \\ \\

Le sumo uno positivo de los restantes: \\ \\
\begin{tabular}{c c c}
\texttt{(7744 + AB68 + 9879 + 5499 + 88BD)}        & \texttt{= 1001100001111011} \\
\texttt{6788}                                      & \texttt{= 0110011110001000} \\
\texttt{(7744 + AB68 + 9879 + 5499 + 88BD + 6788)} & \texttt{= 0000000000000011}
\end{tabular} \\ \\



Una respuesta seria: 7744 + AB68 + 9879 + 5499 + 88BD + 6788. \\
En ese orden no nos da overflow y con 16 digitos vemos que nos da igual a lo que el enunciado indica que es 3.

\end{document}

