
\documentclass[12pt]{article}

% Packages
\usepackage[spanish,activeacute,es-tabla]{babel}
\usepackage{hyperref}
\usepackage{sectsty}
\usepackage{enumitem}
\usepackage{amsmath}

\sectionfont{\normalsize}

\title{Sistemas Digitales - 2da Guia Practica - Lógica circuital}
\author{Román López Garro}

\begin{document}
\maketitle

\begin{abstract}
    Ejercicios resueltos de la segunda guía: Álgebra de Boole, circuitos combinatorios y circuitos secuenciales..

\begin{enumerate}[left=0pt, label=\alph*), leftmargin=*]
	\item Suma booleana: El \textbf{+} representa el OR.
	\item Producto booleano: El \textbf{*} representa el AND.
	\item La suma directa: $ p \oplus q = ( \bar p.q) + (p.\bar q) $
\end{enumerate}


\end{abstract}

\tableofcontents

\newpage

% -------------------------
% Ejercicio 1
% -------------------------
\section*{Ejercicio 1}
\addcontentsline{toc}{section}{Ejercicio 1}
\phantomsection

\textit{Demostrar si las siguientes equivalencias de fórmulas booleanas son verdaderas o falsas: }

\begin{enumerate}[left=0pt, label=\alph*), leftmargin=*]
	\item $x.z = (x + y).(x + \bar y).(\bar x + z)$
	\item $x \oplus (y.z) = (x \oplus y).(x \oplus z)$ donde se aplica la propiedad distributiva con respecto a $\oplus$

\end{enumerate}

\textbf{Solucion: } \\
                    \\

\begin{enumerate}[left=0pt, label=\alph*), leftmargin=*]
	\item $x.z = (x + y).(x + \bar y).(\bar x + z)$
\end{enumerate}

\begin{align*}
	(x+y).(x+\bar y)(\bar x + z) &= x + (y + \bar y).(\bar x + z) \\
	&= x + 0.(\bar x + z) \\
	&= x + \bar x + x.z \\
	&= 0 + x . z \\ &= x.z
\end{align*}

\begin{enumerate}[left=0pt, label=\alph*), leftmargin=*]
	\item $x \oplus (y.z) = (x \oplus y).(x \oplus z)$ donde se aplica la propiedad distributiva con respecto a $\oplus$
\end{enumerate}

Por un lado tenemos que:
\begin{align*}
	(x \oplus y).(x \oplus z) &= ((\bar x.y) + (x. \bar y)) . ((\bar x .z) + (x. \bar z)) \\
	&= (((\bar x . y) + (x . \bar y)). (\bar x . z)) + (((\bar x .y) + (x . \bar y)).(x. \bar z)) \\
	&= (\bar x + y + z) + (x . \bar y. \bar z)
\end{align*}

Por otro lado tenemos que:
\begin{align*}
	x \oplus (y.z) &= (\bar x . y . z) + (x . \overline{(y.z)})
\end{align*}
luego vemos que son dos expresiones diferentes.


\end{document}

